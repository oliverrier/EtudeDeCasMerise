\documentclass{beamer}
\usepackage[utf8]{inputenc}
\usepackage[french]{babel}
\renewcommand{\thefootnote}{\*} % Pour éviter les numéros moches !

\begin{document}

\begin{frame} % La page de titre.
\title{La Méthode Merise}
\author{FEYDIT Rémi, MAURY Louis, VERRIER Olivier}
\titlepage
\end{frame}

\begin{frame}{La Méthode Merise} % La page de sommaire.
\begin{itemize}
\item C'est \textbf{Quoi} ?
\item C'est de \textbf{Qui} ?
\item Ça vient d'\textbf{Où} ?
\item Depuis \textbf{Quand} ?
\item On l'utilise \textbf{Comment} ?
\item \textbf{Combien} ?
\item \textbf{Pourquoi} ?
\end{itemize}
\end{frame}

\begin{frame}{C'est \textbf{Quoi} ?}
Merise est une méthode d’analyse, de conception et de gestion de projet informatique.
\footnote{La Méthode Merise}
\end{frame}

\begin{frame}{C'est de \textbf{Qui} ?}
Elle a été créée par \textsl{René Colletti}, \textsl{Arnold Rochfeld} et \textsl{Hubert Tardieu}.
\footnote{La Méthode Merise}
\end{frame}

\begin{frame}{Ça vient d'\textbf{Où} ?}
La méthode Merise vient de France et est utilisée au sein d’organisation et de société.
\footnote{La Méthode Merise}
\end{frame}

\begin{frame}{Depuis \textbf{Quand} ?}
Elle est apparue dans le début des années 70 et a connu une forte utilisation dans les années 70-80.
\footnote{La Méthode Merise}
\end{frame}

\begin{frame}{On l'utilise \textbf{Comment} ?}
Avec divers logiciels tels que \texttt{draw.io}, \texttt{JMerise}, \texttt{JMCT}, \texttt{Analyse SI}
\footnote{La Méthode Merise}
\end{frame}

\begin{frame}{\textbf{Combien} ?}
\begin{itemize}
\item De gens l'utilisent ?\\
Plus de 104 000 personnes sans compter le nombre de gens dans les grandes administrations publiques ou privées.
\item Ça coûte ?\\
Les prix peuvent varier suivant la demande du client, le temps que ça va prendre, etc...
\end{itemize}
\footnote{La méthode Merise}
\end{frame}

\begin{frame}{\textbf{Pourquoi} ?}
La méthode Merise est utilisée pour les projets internes aux organisations pour l’informatisation massive.
Elle permet la représentation de future base de données pour une application physique.
\footnote{La Méthode Merise}
\end{frame}

\begin{frame}
\title{La Méthode Merise} % Page de fin
\author{FEYDIT Rémi, MAURY Louis, VERRIER Olivier}
\centering{Merci de nous avoir écouté.}
\titlepage
\end{frame}

\end{document}